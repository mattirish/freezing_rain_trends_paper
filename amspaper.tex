%% Version 4.3.2, 25 August 2014
%
%%%%%%%%%%%%%%%%%%%%%%%%%%%%%%%%%%%%%%%%%%%%%%%%%%%%%%%%%%%%%%%%%%%%%%
% Template.tex --  LaTeX-based template for submissions to the 
% American Meteorological Society[RBR1]
%
% Template developed by Amy Hendrickson, 2013, TeXnology Inc., 
% amyh@texnology.com, http://www.texnology.com
% following earlier work by Brian Papa, American Meteorological Society
%
% Email questions to latex@ametsoc.org.
%
%%%%%%%%%%%%%%%%%%%%%%%%%%%%%%%%%%%%%%%%%%%%%%%%%%%%%%%%%%%%%%%%%%%%%
% PREAMBLE
%%%%%%%%%%%%%%%%%%%%%%%%%%%%%%%%%%%%%%%%%%%%%%%%%%%%%%%%%%%%%%%%%%%%%

%% Start with one of the following:
% DOUBLE-SPACED VERSION FOR SUBMISSION TO THE AMS
% \documentclass{ametsoc}

% TWO-COLUMN JOURNAL PAGE LAYOUT---FOR AUTHOR USE ONLY
\documentclass[twocol]{ametsoc}

%%%%%%%%%%%%%%%%%%%%%%%%%%%%%%%%
%%% To be entered only if twocol option is used

\journal{jcli}

%  Please choose a journal abbreviation to use above from the following list:
% 
%   jamc     (Journal of Applied Meteorology and Climatology)
%   jtech     (Journal of Atmospheric and Oceanic Technology)
%   jhm      (Journal of Hydrometeorology)
%   jpo     (Journal of Physical Oceanography)
%   jas      (Journal of Atmospheric Sciences)	
%   jcli      (Journal of Climate)
%   mwr      (Monthly Weather Review)
%   wcas      (Weather, Climate, and Society)
%   waf       (Weather and Forecasting)
%   bams (Bulletin of the American Meteorological Society)
%   ei    (Earth Interactions)

%%%%%%%%%%%%%%%%%%%%%%%%%%%%%%%%
%Citations should be of the form ``author year''  not ``author, year''
\bibpunct{(}{)}{;}{a}{}{,}

%%%%%%%%%%%%%%%%%%%%%%%%%%%%%%%%%%%%%%%%%%%%%%%%%%%%%%%%%%%%%%%%%%
%% Matt's added packages: (idk if I'm allowed to do this in my submission but I think so)
\usepackage{amsmath}
\usepackage{graphicx}
\usepackage[colorinlistoftodos]{todonotes} %% For comments that will show in the pdf
\usepackage{gensymb} %%for degree symbols mainly



%%%%%%%%%%%%%%%%%%%%%%%%%%%%%%%%

%%% To be entered by author:

%% May use \\ to break lines in title:

%% \title{Trends in Ice Storms throughout the Great Lakes Region}
\title{Recent Changes in Freezing Rain and Ice Storms\\ throughout the Great Lakes Region}

%%% Enter authors' names, as you see in this example:
%%% Use \correspondingauthor{} and \thanks{Current Affiliation:...}
%%% immediately following the appropriate author.
%%%
%%% Note that the \correspondingauthor{} command is NECESSARY.
%%% The \thanks{} commands are OPTIONAL.

    %\authors{Author One\correspondingauthor{Author One, 
    % American Meteorological Society, 
    % 45 Beacon St., Boston, MA 02108.}
% and Author Two\thanks{Current affiliation: American Meteorological Society, 
    % 45 Beacon St., Boston, MA 02108.}}

\authors{Matt Irish\correspondingauthor{Matt Irish, National Renewable Energy Laboratory, 15013 Denver West Parkway, Golden, CO 80401.}}\thanks{Work done while at the Department of Climate and Spaces and Engineering, University of Michigan, Ann Arbor, Michigan.}

\affiliation{ National Renewable Energy Laboratory, Golden, Colorado}

\email{mairish@umich.edu}

%% If appropriate, add additional authors, different affiliations:
    %\extraauthor{Extra Author}
    %\extraaffil{Affiliation, City, State/Province, Country}
\extraauthor{Richard B. Rood}
\extraaffil{Department of Climate and Space Sciences and Engineering, University of Michigan, Ann Arbor, Michigan}
\extraauthor{William J. Baule}
\extraaffil{Department of Geography, Environment, and Spatial Sciences, Michigan State University, East Lansing, Michigan}

%%%%%%%%%%%%%%%%%%%%%%%%%%%%%%%%%%%%%%%%%%%%%%%%%%%%%%%%%%%%%%%%%%%%%
% ABSTRACT[RBR3]
% Abstracts should not exceed 250 words in length!
\abstract{Despite the expensive and deadly impacts that freezing rain events have upon major population centers in the Great Lakes region of North America and the region's status as a center of freezing rain activity, little research has been devoted to better understanding the spatial and temporal distribution of these events and their recent trends, especially in regard to anthropogenic climate change. This work extends the most recent Great Lakes freezing rain climatology (1976--1990) to analyze trends and their underlying dynamics by making use of observations made through 2014. To isolate regional trends and determine changes in the synoptic-scale processes driving them, a k-means objective clustering algorithm is used first to create three archetypal synoptic weather patterns associated with freezing rain events and then to group weather stations into regions subject to unique synoptic conditions during freezing rain events. A northward shift in freezing rain frequency is evident across the Great Lakes on an annual basis, accompanied by a seasonal trend toward more frequent freezing rain occurrence in January and few incidences of freezing rain in November and March. The intensity of freezing rain events has increased in all regions except South-Central Canada since automated freezing rain instrumentation began in 1995. The largest regional trend has been a **\% decrease (+1.9 hrs yr$^{-1}$ decade$^{-1}$) in hourly freezing rain observations in the Appalachian Range since 1976 despite no discernible change in total precipitation. South-Central Canada has seen an increase of **\% (-**hours decade$^{-1}$) in the context of a slight increase in overall precipitation, and in the New York City area an increase of **\% (-1.3 hrs yr$^{-1}$ decade$^{-1}$) has been observed despite a trend toward less overall precipitation each year. The low-pressure centers of each of the three synoptic weather patterns have migrated northward roughly ~200-500 km between the first and second half of the 1979--2014 period, consistent with warmer temperatures and northward-moving extratropical cyclone tracks. Observed trends have generally matched or outpaced model projections.}

\begin{document}
%% Necessary!
\maketitle
%%%%%%%%%%%%%%%%%%%%%%%%%%%%%%%%%%%%%%%%%%%%%%%%%%%%%%%%%%%%%%%%%%%%%
% MAIN BODY OF PAPER
%%%%%%%%%%%%%%%%%%%%%%%%%%%%%%%%%%%%%%%%%%%%%%%%%%%%%%%%%%%%%%%%%%%%%
%
\section{Introduction}
The relatively mild temperatures present during ice storms often leave them neglected in discussions of extreme weather events; however, the detrimental impacts of accumulated glaze ice from freezing rain events often compare with those of tornadoes, severe thunderstorms, and in a few severe events, hurricanes \citep{lott2006tracking}. In North America, ice storms occur most frequently in the upper U.S. Midwest, northeastern U.S., southeastern Canada, and in some locations throughout the Pacific Northwest \citep{changnon2003temporal}. In the northeastern U.S., adjacent to the St. Lawrence river valley, freezing rain was present during roughly 5--7 days each year throughout the middle of the 20th century \citep{changnon2003temporal}. In Pennsylvania, New York, and New England, the probability during a single ice storm glazing of more than 0.64 cm (0.25 inches, the standard used by the National Weather Service to roughly define the minimum ice accretion constituting an ice storm) during a given year was 88\% \citep{nwsglossary,tattelman1973estimated}. 

Over the past several decades, single storms have caused billions of dollars of damage and taken dozens of lives, with a yearly average cost of \$193 million and 26 deaths attributed to ice and sleet events in the latter half of the 20th century \citep{lott2006tracking, changnon2006severe, ncei2019storm}. The most obvious impacts of these events are on transportation, but power outages due to heavy icing on power lines are among the greatest challenges. Injuries and deaths from vehicular accidents, falling trees, and downed power lines are common, and deaths have resulted from carbon monoxide inhalation due to the use of poorly ventilated generators during power outages \citep{daley2000outbreak}. Ice storms can have power generation impacts as well, with wind turbines forced offline due to blade icing sometimes for days and even up to two weeks in some regions \citep{davis2014forecast}. Data on the costs to generators and ratepayers of such shutdowns have not been analyzed and many system operators do not collect data on wind curtailment. The New England Independent System Operator, the organization responsible for operating that region's power system, reported that shutdowns due to blade icing sometimes account for a substantial portion of the total yearly wind curtailment in the region \citep{bird2014wind}. As development of new wind energy projects accelerates, this makes understanding recent trends in freezing rain occurrence particularly important for financial projections, power system reliability and resilience planning studies, and safety considerations during siting. 

The damage to natural systems by the presence of glaze ice has also been explored, mainly in regard to forest management and damage to vegetation that wildlife relies on for food and shelter \citep{pellikka2000modelling}.

\missingfigure{Will be a cartoon illustration of freezing rain impacts. It's not necessary but I [RBR6]haven't found a good one yet so I thought it might be nice for people to be able to use.}

Ice storms generally occur when the surface temperature is just below freezing. Events often form latitudinal bands across the regions they affect, bounded by snow to the north and rain to the south. These bands are expected to shift northward as the climate continues to warm \citep{cheng2011possible,lambert2011simulated}. It is intuitive how this would occur as the 0\degree C isotherm migrates poleward, but this simplistic notion is complicated by the fact that freezing rain events are triggered under several different synoptic conditions and are heavily influenced by both topography and proximity to large bodies of water such as the Great Lakes and the Atlantic Ocean. 

In an effort to understand regional differences in freezing rain formation, \citet{rauber2001synoptic} carried out a manual classification of 411 ice storms occurring east of the Rocky Mountains from 1970-1994 and identified seven archetypal categories of freezing rain storms. These classifications were later consolidated by \citet{erfani2012automated} into three groups by a k-means objective typing algorithm and statistical analysis carried out on the same storms. 

Looking forward into the 21st century, \citet{lambert2011simulated} applied a precipitation typing algorithm to climate model simulations of a moderate warming scenario for 2081-2100 and found that freezing rain events are expected to shift poleward with modest increases in freezing rain incidence to the north and substantial decreases to the south. \citet{cheng2011possible} came to similar conclusions using statistical downscaling and synoptic weather typing of climate projections to predict substantial increases in freezing rain frequency throughout southeastern Canada during the colder winter months of January and February with a progressively greater effect from south to north. Decreases were projected in the fall and warmer winter months. The two studies, however, focused on slightly different regions and came to differing conclusions about whether seasonal increases overwhelm the decreases on an annually-averaged basis.

%%\subsection{Objective}
\citet{groisman2016recent} provided the first evidence supporting the hypothesis that temporal changes in freezing rain distributions have been observed. Their analysis focused on North America and Europe as a whole and the analysis of trends was limited to concluding that there have been increases in the arctic region of North America and decreases in the southeastern U.S. Most ice storms and the greatest impacts on humans, however, are concentrated between those two extremes, in the mid-latitudes.

The most recent freezing rain climatology to focus on the Great Lakes region was published in 2000, made use of observations from 1976 to 1990 \citep{cortinas2000climatology}, and did not analyze trends. The purpose of this paper is to build upon that climatology by analyzing trends in the frequency, intensity, and duration of freezing rain events through 2014 and then to identify the underlying changes in the weather systems that bring them about. These changes in climate dynamics are compared with those previously linked to human emissions of greenhouse gases in the scientific literature to investigate how the observed trends fit into the larger frame of anthropogenic climate change.


\section{Data}
The study area was defined similarly to that of \citet{cortinas2000climatology}, but reaching further eastward, from 96\degree W to 72\degree W and 38\degree N to 50\degree N. Data were collected for the U.S. states of Illinois, Indiana, Iowa, Michigan, Minnesota, New York, Ohio, Pennsylvania, and Wisconsin, as well as the Canadian provinces of Ontario and Quebec (Fig. \ref{fig:trendmap}). 

The infrequent nature of freezing rain events suggests a signal-to-noise ratio close to one, and the use of weather station observations for trend analysis is made difficult by instrumentation changes in the observational record and the highly localized nature of freezing rain formation.  The observation of freezing rain uses hourly surface observations from the Integrated Surface Database (ISD), maintained by the National Centers for Environmental Data \citep{smith2011integrated}. Observations span from 1976 to 2014 and include data from human observers, the Automated Weather Observing System (AWOS), the Automated Surface Observing System (ASOS), and automated stations managed by Environment and Climate Change Canada (ECCC).

ISD data were filtered to remove special (i.e. non-standard) observations and to normalize the number of observations to one per hour when observed weather types were present. A standard of 90\% data completeness was adopted for this work, compared with \citet{cortinas2000climatology}, which excluded any location with less than 80\% present weather reports. Some locations were excluded for which manual observers were not present throughout the night in the winter months up until the mid-1990s though the overall threshold for data completeness was met. Despite efforts in quality control, issues remain with intermittent manual observation during some hours in the first two decades of observations. After performing this quality control, 97 stations out of an initial 121 met the criteria to be used in the analysis. An average of 97.4\% of hours each year were accounted for at the included stations.

An important caveat regarding the use of the ISD dataset is that it spans a transition in observing techniques from a period involving strictly manual freezing rain observations (1976-1994) to the automation of reporting with new freezing rain instruments beginning in 1995 \citep{ramsay1995status,asos1998}. Inspection of the data revealed that the change from manned to automated observations did not occur at the same time across all locations. Comparison of pre- and post-transition freezing rain frequencies showed that any bias present does not appear to overwhelm any temporal or spatial trends, and a change-point analysis did not identify any change points in 1995 a-priori \todo[inline]{Elaborate here}.

Given the value of analyzing freezing rain based on synoptic weather conditions, synoptic-scale weather variables were examined using the North American Regional Reanalysis (NARR) produced at the National Centers for Environmental Prediction \citep{mesinger2006north}. The variables of interest were mean surface level (MSL) pressure, 850 mb geopotential heights, and air temperature at 2 m. The NARR was chosen for its consistency of temporal assimilation and high resolution (32 km at 3-hour intervals). The reanalysis begins in 1979, leaving observations from 1976-1978 outside the range that can be investigated using NARR.


\section{Methods}
A straightforward trend analysis only requires the total number of hours of freezing rain recorded over a time interval. To gain a fuller picture of how the climatology of ice storms may have changed, the definition of a freezing rain event is necessary. The U.S. National Weather Service terms an ice storm an event "when damaging accumulations of ice are expected during freezing rain situations," adding that the term generally refers to accumulations of 0.64 cm (0.25 in.) and greater \citep{nwsglossary}. Although ice storms are the topic of interest in this work, such a criterion is inappropriate for this analysis since ice accretion occurs at different rates on different surfaces and under different conditions, and estimation of radial accretion of ice on a cylindrical surface using variables such as wind and dewpoint temperature has shown inconsistent results \todo[inline]{edit this bit and give reasons for not doing model. mainly that we don't have access to sensor frequency data.}. 

Therefore, freezing rain events are used as a proxy for ice storms in an analysis of the dynamics underlying observed changes in freezing rain frequency. A freezing rain event is defined as having begun at the first hourly report of freezing rain anywhere in the region and continuing until at least three hours has elapsed without any further reports of freezing rain across the domain of study. For purposes of classifying freezing rain events, events with less than four reports of freezing rain were excluded to eliminate hyper-localized reports; however, all observations of freezing rain are included in the initial frequency trend analysis. Although the total number of events observed was highly sensitive to the choice of minimum reported hours of freezing rain and the maximum number of intervening hours between freezing rain reports, the results of the analysis of synoptic weather conditions present during storms was not. \todo[inline]{report percentage of hours that weren't included in ice storms.}

\subsection{Analysis of trends}
A mix of parametric and non-parametric methods were used in the trend analysis depending on the data type being studied and the validity of assumptions necessary to use parametric tests. Changes in the intensity of freezing rain were analyzed using standard least-squares linear regression. The existence of trends in the seasonal and annual frequency of freezing rain reports was tested for using a modified form of the non-parametric seasonal Mann-Kendall test for a monotonic trend and the Theil-Sen method for estimaing a linear slope. Both are techniques to assess monotonic trends \citep{chandler2011statistical}. Though they offer less statistical power than parametric methods, such tests were chosen because the time series of freezing rain reports are positively skewed, with a small number of long-lasting, sprawling storms relative to the number of shorter, more lozalized events. This was confirmed by plotting histograms of the observations.\todo[inline]{add reference to histogram either in paper or in appendix}

\subsubsection{Modified seasonal Mann-Kendall test for monotonic trend}
The Mann-Kendall test assesses the relative magnitude (rank) of observations rather than their absolute values. The null hypothesis of the test is that the data are randomly ordered as opposed to exhibiting a monotonic trend in time. The test is carried out for the sequence of both annual and monthly total hours of freezing rain reported at each location, $y_1,\ldots,y_T$. The differences of each permutation of pairs $\{d(t_1,t_2)=y_{t2}-y_{t1}:t_2>t_1\}$ is calculated for all $T$ observations and then assigned values of $+1$ for positive differences, zero for ties, and $-1$ for negative differences. The test statistic for the original Mann-Kendall method, $S$, is the sum of all these values \todo[inline]{clarify equations here WRT RBR13:

\[\sum_{t_1=1}^{T-1}\sum_{t_2=t_1=1}^T\text{sgn}[d(t_1,t_2)],\] where\\

\[\text{sgn}[d(t_1,t_2)]=\begin{cases} 1 & \text{if } d(t_1,t_2)>0\\ 0 & \text{if } d(t_1,t_2)=0\\ -1 & \text{if } d(t_1,t_2)<0.\\ \end{cases}\]

So $S$ can take any integer value from $-T(T-1)/2$ to $T(T-1)/2$. $S$ is proportional to the nonparametric Kendall's rank correlation coefficient, $\tau$, a measure of the rank correlation between the annual or monthly freezing rain reports and time with range $\tau=\pm1$:

\[\tau=\frac{2S}{T(T-1)}\]

If $S\approx0$, then $\tau\approx0$ and the null hypothesis of no trend is accepted, whereas if  $S\approx\pm T(T-1)/2$, $\tau\approx\pm1$ and the null hypothesis is rejected. 

The test described above is expanded into a multivariate Mann-Kendall trend test to explore monthly trends and to accommodate serial dependence (i.e. nonzero correlation of total freezing rain reports across subsequent time intervals) within the dataset. This involves expanding the time series into a $T\text{ by }p$ matrix, where the number of seasons $p=12$ (meaning that the term "seasonal" in this application is referring to months rather than the four seasons of spring, summer, autumn, and winter). The null hypothesis is that for each of the $p$ seasons, the $T$ observations are randomly ordered. The calculation of the new test statistic and its variance is too involved to fully describe here, but \citet{hirsch1984nonparametric} (pg. 728) detail the derivation, which focuses on summing contributions from individual seasons \todo[inline]{fix this paragraph and all references to "seasonal" and just call it monthly. Say "although often referred to as monthly,"}.

The original Mann-Kendall test and its monthly expansion assume the time series is serially independent, in which case the null distribution can be approximated by a normal distribution with mean zero and an analytically-defined variance \citep{kendall1955rank}. But this is not the case for some of the freezing rain time series analyzed here, so the test had to be modified with a different variance. This is akin to considering the degrees of freedom in parametric significance tests. Serial dependence in the dataset was tested for by calculating the autocorrelation coefficient (defined here as the non-parametric Spearman correlation coefficient for a time lag of one time interval [e.g. one month or one year]) at each location. The median autocorrelation across the 97 stations was $r_s = 0.0406$.  There were, however, several stations with statistically significant ($p<0.05$) autocorrelations as high as $r_s = 0.5291$. The required modification to the monthly Mann-Kendall test further reduces statistical power as compared with parametric methods, but was considered necessary to maintain statistical rigor. The modified variance calculation was implemented in MATLAB according to the method described by \citet{hirsch1984nonparametric}. \todo[inline]{Prob don't do this, but read the reference to see how much correlation is necessary and edit this to reflect it.}

\subsubsection{Theil-Sen slope estimator}
In the case that the Mann-Kendall test indicates a monotonic trend exists, the trend is assumed to be linear and the slope calculated. The Theil-Sen slope estimator is a method commonly paired with Mann-Kendall; it is much more accurate than simple linear regression for non-normal data and is robust in the presence of outliers. The slope is defined as the median of the slopes calculated between every permutation of points in the time series, $\{d(t_1,t_2)/[t_2-t_1]:t_2>t_1\}$. As a linear slope estimation, it can be seen as the "typical" index of change over a unit of time \citep{chandler2011statistical}. The confidence interval is then calculated by computing an inverse normal cumulative density function \todo[inline]{expound a bit on that, per ricky comment}.

\subsection{Classification of regions and analysis of trends in meteorological forcing}
An initial inspection of the temporal and spatial evolution of freezing rain---diurnally, monthly and annually as well as across the domain and at a higher, regional resolution---was carried out to look for clues regarding changes in physical mechanisms.  This is difficult because of the rare and highly local nature of the events---often just a few large storms every year or so obfuscate any slower underlying changes.

As mentioned in the introduction, synoptic conditions during ice storms can be broadly categorized into several distinct archetypal patterns, varying from those caused by extratropical cyclone-anticyclone interactions (mainly in the Midwest to southeastern Canada) to those caused mainly by topography (cold air damming and trapping throughout the Northeast Appalachian Range and along its eastern slopes). By analyzing these synoptic conditions and then classifying stations into regional groups that share similar conditions and freezing rain frequency distributions, we identify regional similarities in the dynamics driving freezing rain events.

A synoptic weather typing and clustering procedure is carried out as follows:

\begin{enumerate}
\item
For each event identified across the study region from 1976 to 2014, two anomaly maps were obtained---one of MSL pressure and one of 850 mb geopotential height---from the three-hourly NARR output closest to the middle of event occurrence. Anomalies were determined by subtracting the monthly climatology of the entire 1979--2014 period.
\item
The k-means clustering algorithm (with $k=3$, in accordance with \citet{erfani2012automated} and verified using silhouette plots) was applied to the anomaly maps and the resulting centroid maps were constructed. These centroid maps represent the mean pressure anomalies for each of the three storm patterns.
\begin{itemize}
\item
The MSL pressure and 850 mb height anomaly maps for each event were reshaped into vectors and concatenated, creating a matrix with each row representing an event and each column representing a spatial grid cell. k-means clustering was chosen since it performed the best in \citet{erfani2012automated}'s freezing rain synoptic typing validation, outperforming hierarchical and average linkage methods.\todo[inline]{how did it "outperform"? Explain. otherwise, this is a bit informal.}
\item
Events were classified into one of three clusters and three centroid maps of MSL pressure anomaly and 850 mb height anomaly were created and analyzed.
\item
The clusters were compared to the archetypal patterns manually classified by \citet{rauber2001synoptic} for physical intuition.
\end{itemize}
\end{enumerate}

To link freezing rain trends with possible changes in the meteorological forcing of events, the steps above were repeated, dividing the time record into two spans. The new centroid maps were compared to old ones to identify any changes in synoptic conditions. In addition to rerunning the k-means algorithm to create new clusters, the events in the 1997--2014 period were also classified into the original clusters created from the 1979--2014 dataset by assigning each one according to the minimum Euclidean distance between the event and the centroids. 

Producing separate cluster results for each of the two periods allowed analysis of how underlying modes themselves may have changed, and binning the more recent events to the original clusters trained on the first period made it possible to investigate how the relative importance of different archetypal patterns in freezing rain formation may have shifted.

\section{Results}

\subsection{Observed trends at individual stations and domain-wide}

\todo[inline]{ADD: (-1.5$\pm$2.1\degree C [1$\sigma$] in the Great Lakes and Northeast during an event)}

Figure \ref{fig:trendmap} depicts linear trends in total annual hours of freezing rain as calculated using the Theil-Sen method at all 97 stations. The values are shown as percentage change from a 1976--1985 baseline to the 2014 total hours of freezing rain expected by the trend line. As shown, the modified Mann-Kendall test indicated the existence of monotonic annual trends at 10 locations at the $p<0.05$ significance level. While there appears to be a general decrease in annual hours of freezing rain in the south of the region and an increase to the north in south-central Canada, much of the region in the mid-latitudes has not experienced a consistent annualized trend over the past 39 years. This is intuitive, since expected decreases in the "shoulder" fall and spring months and increases during the mid-winter months may be expected to have competing effects on the overall annual freezing rain frequency.

\begin{figure*}
\centering
\includegraphics[width=1.0\textwidth]{trendmap.png}
\caption{\label{fig:trendmap}Map of the region of interest showing trends in total annual number of hours of freezing rain reported computed using Theil-Sen method. Locations are scaled by confidence level ($1 - p$) [with plus signs denoting stations with small circles, for clarity]. Xs denote locations where modified Mann-Kendall test indicates a monotonic trend at the $p<0.05$ level. Note: baseline is first decade of observations.}
\end{figure*}
\todo[size=\small]{re: figure: get rid of black circles. Add baseline map + map showing abs trends in hours and get topo for all.} 

Figure \ref{fig:trendmap} indicates distinct regional trends. Stations along the northeastern Appalachian Mountains, the Catskills, the Adirondacks, for example, which frequently experience topography-forced events, have experienced a consistent decrease in freezing rain event frequency, along with stations near the eastern shores of Lakes Erie and Ontario. Stations centered on Long Island, on the other hand, indicate substantial increases.  

\begin{figure*}
\centering
\includegraphics[width=.9\textwidth]{Seasonal.PNG}
\caption{\label{fig:seasonal} Comparison of seasonal frequency of freezing rain events between the first 21 years in the study period and the last 18 (to better coincide with the NARR-based dynamics analysis below). Red lines demarcate mean values, boxes signify 25-75\% percentiles, whiskers inner fences, and red plus signs indicate outliers. Table lists $p$-values calculated using Mann-Whitney rank sum method to test for difference in median value.}
\end{figure*}
\todo[size=\small]{re: figure: add Mann-Whitney p-values in their own row.}

Monthly trends in freezing rain frequency averaged across the entire Great Lakes region reveal a significant monotonic trend only for March \todo[inline]{update after removing modification to MK}. A weaker hypothesis is that the average freezing rain frequency for some months has changed.\todo[inline]{[RBR25]I think this sentence needs changing.  Transition from previous sentence made more clear, perhaps?  We have, also, moved to the split time period.  This paragraph is one of our key results and clarity in transition is needed.   Lead with Mann-Whitney. (reference). Mann-Whitney described earlier, we use it.  Don�t just have it leap out.  Weaker than what? THIS paragraph is important.  Unpack it and describe it. (Matt comment: agreed, this part is very bad)} This was tested for using the Mann-Whitney test, a non-parametric counterpart to the two-sample t-test that determines whether the medians of two series are different from one another. Figure \ref{fig:seasonal} shows the frequency … and the $p$-values calculated for each month after computing the Mann-Whitney test, which independently compares the rank of all entries of the two time series for the 1976--1996 period and the 1997--2014 period. As shown, the median duration of freezing rain in the months of November and March has changed, along with the mid-winter month of January. The box-and-whisker plots show the "tightening up" of the winter distribution of freezing rain occurrence, as suggested by prior studies like \citet{cheng2011possible}, has likely begun to occur. However, all that can be said to be robust from this analysis of seasonality is that the median freezing rain frequency has changed in recent years for the months of November, January, and March, and that there has been a robust region-wide decrease in freezing rain occurrence during March.

\begin{figure*}
\centering
\includegraphics[width=1\textwidth]{Cluster_Centroids.png}
\caption{\label{fig:centroids} Centroid maps of cluster analysis  used to create three archetypal storm patterns. Solid lines refer to MSL pressure anomalies and dotted lines demarcate 850 mb geopotential height anomalies. Color bar shows anomalies, not absolute pressures.}
\end{figure*}

\subsection{Classification of storms into three archetypal patterns}
The k-means clustering algorithm used to group storms into three archetypal patterns resulted in the centroid pressure and geopotential height anomaly maps shown in Figure \ref{fig:centroids}. \citet{erfani2012automated} produced average anomaly maps of Rauber's manually selected groups that were used here for comparison. Table \ref{archetypalpatterns} summarizes these comparisons and key statistics for each cluster.

The first cluster (Figure \ref{fig:centroids}A) is similar to Rauber's pattern B, describing freezing rain events that occur at the occlusion sector of warm fronts, as well as Rauber's pattern G, describing cold air trapping. The former occurs just north of the 0\degree C surface isotherm when warm air overruns a subfreezing surface layer. Cold air trapping occurs when approaching continental cyclones advect warm air over the Appalachians, trapping cold air in valleys at the surface and leading to freezing rain. Both MSL pressure and 850 mb anomalies closely resemble those plotted by \citet{rauber2001synoptic}, where  8.1\% of freezing rain events between 1970-1994 were classified as being caused by these conditions. \todo[size=\small]{compare with \% from this study}

The second cluster (Figure \ref{fig:centroids}B) represents the extratropical cyclone-forced storm that most often affects the Midwestern U.S. and especially its southern region (Rauber pattern C). Storms that take on this pattern were identified by Rauber to be some of the longest in duration and most damaging throughout the Midwestern U.S., due to their high-pressure gradients and the higher winds that result. \todo[size=\small]{compare with \% from this study}

The third cluster (Figure \ref{fig:centroids}C) is similar to Rauber's patterns D and E. The D group refers to systems that produce freezing rain in the western quadrant of arctic high-pressure anomalies interacting with warmer air from the Southeast. The E pattern describes cold air damming, synoptic conditions which trigger freezing rain when arctic air masses are dammed against the eastern slopes of the Appalachian range and warmer air moving in as a result of easterly flow from the Atlantic overtakes it. Rauber identified this archetypal pattern as contributing to 5.7\% of all storms and many of the longest, with a quarter of events lasting longer than 12 hours.\todo[size=\small]{compare with \% from this study} Freezing rain from these systems would be expected in southern Quebec and Ontario, in the freezing rain hotspots of Ottawa and Montreal and throughout the Appalachian Range. \todo[inline]{rework this with new clsuters andadd discussion of table data}

\begin{table*}
\label{archetypalpatterns}
\caption{Archetypal patterns of synoptic conditions during freezing rain events as determined through k-means clustering.}
\begin{tabular}{p{0.05\linewidth}p{0.3\linewidth}p{0.1\linewidth}p{0.1\linewidth}p{0.1\linewidth}p{0.1\linewidth}p{0.05\linewidth}}
\topline
Cluster & Description                 & Similar to Rauber patterns & No. of events & Mean duration (hrs.) & \% of reports "light" intensity &  \\ 
\midline
1       & Warm front occlusion and cold air trapping       & B and G      &           &                                          &                                                     &  \\
2       & Cyclone/Anticyclone                              & C            &           &                                          &                                                     &  \\
3       & Western quadrant of arctic high/cold air damming & D and E      &           &                                          &                                                     &  \\
\botline
\end{tabular}
\end{table*}


\subsection{Changes in meteorological forcing}
Changes in meteorological forcing were investigated by separately clustering the first and second half of the 36-year time period of NARR coverage and examining the changes in centroid anomaly maps from the former to the  latter half of observations. While there are close similarities with the clusters identified in the analysis of regions above, several notable differences are observed for the clusters as computed for the two separate time periods and for the clusters as trained upon the first half of the time period later in this section. As such, the clusters are numbered to highlight that distinction. The results are depcited in Figure \ref{fig:clusters}.

\citet{erfani2012automated} found that three clusters were appropriate to describe the significantly different archetypal patterns causing freezing rain by analyzing wind and precipitation variances for different values of $k$. Their analysis, however, carried out two different clustering algorithms, one for MSL pressure and one for pressure aloft, whereas here both are considered in the same clustering process. 
Comparing with the archetypal patterns set out by Rauber and the clusters identified by Erfani, Cluster 1 indicates a cyclone/anticyclone setup (Rauber pattern C). The northward shift and apparent intensification of the cyclone is striking. This is consistent with observations dating back to roughly 2000 of a northward shift in cyclone tracks, thought to be forced at least in part by increasing temperatures due to enhanced mixing ratios of atmospheric CO$_2$ \citep{mccabe2001trends}. While becoming less frequent, these cyclones have also become more intense. These historically have also been the longest-lasting events, with 25\% of these events lasting more than 24 hours \citep{rauber2001synoptic}. 

Little change seems to be observed in Clusters 2 and 3 across the two periods. Cluster 2 encapsulates Rauber patterns F and G, the topography-forced cold-air damming and trapping conditions. A similar cluster was created by Erfani. Cluster 3, which is similar to Rauber's pattern B, characterizing the classical formation of freezing rain caused at the warm front and occlusion sector of cyclones, appears to have changed little as well.\todo[size=\small]{take a closer look. There are some clear changes, esp. with northward movement of L pressure center in 2 and an eastward push of the low pressure system in 3}

\begin{figure*}
\centering
\includegraphics[width=\textwidth]{Clusters.PNG}
\caption{\label{fig:clusters} Freezing rain event archetypal patterns created using k-means algorithm with $k=3$. Color ramp and solid lines refer to MSL pressure anomalies in mb, while labeled dotted lines refer to 850 mb geopotential height anomalies in m.}
\end{figure*}

\begin{figure}
\centering
\includegraphics[width=0.5\textwidth]{Storm_Pattern_Change.png}
\caption{\label{fig:clusterchange} Timeline showing all events as classified into each of three clusters as calculated for the 1979--1996 period (top) and change in classification of events from the 1979--1996 period to the 1997--2014 period where the latter period is classified according to the centroids trained on the earlier period (bottom).}
\end{figure}

To investigate any changes in the contribution of each archetypal pattern to annual freezing rain totals, the prevalence of events in the 1979--1996 period is compared with that of events from the 1997--2014 period after each event has been assigned to a centroid computed from the earlier data. Figure \ref{fig:clusterchange} compares the number of freezing rain events associated with each pattern between the first and second half of the study period. 586 events were identified during the first 18-year period, and 625 events were identified during the second period, an overall increase of 6.6\%. The number of freezing rain events under Clusters 1 and 2 \todo[size=\small]{Get consistent with either naming clusters by numbers or not} rose substantially, indicating an interdecadal increase in the number of often lengthy extratropical cyclone/anticyclone-forced events (25.2\%) as well as those forced by topography in the Appalachian Mountains and eastward (13.8\%). The number of storms caused in the classic warm front/occlusion zone setup decreased 13.6\%.  The number of storms caused in the warm front/occlusion zone pattern decreased 13.6\%. 

The most salient conclusion looking at this dynamical analysis as a whole was that the largest portion of increases in freezing rain occurrence stems from an archetypal synoptic weather pattern that has led to the longest and most dangerous events, as well as being the one that is most clearly tied to human influence. This fits into the observed pattern of winter storms having become larger in spatial scale and intensity over time \citep{changnon2007catastrophic}. 

\section{Discussion and conclusion}
Linear trends were generally consistent with what has been predicted in the few studies on future freezing rain activity done to date, with the northern regions showing faster-than-expected increases in freezing rain frequency. Changes in seasonal freezing rain occurrence consistent with predictions are robust at a region-wide scale, even if the magnitude of the seasonal trends are not. Several regional trends were also identified that invite further scrutiny and could be of practical use. For example, decreases in freezing rain frequency throughout the Appalachian Mountains and westward to the eastern shores of Lakes Erie and Ontario are among the most prominent trends despite not having been predicted to date.

The division of the time period into halves for k-means clustering analysis provided insight into recent shifts in freezing rain dynamics. The northward shift of the low pressure system depicted by Cluster 1 is a compelling sign that increased temperatures may be pushing the typical freezing rain "belt" northward. With insight from the body of climate literature connecting northward movement of North American mid-latitude storm tracks with anthropogenic influence, it appears likely that the freezing rain climatology has changed under human influence. \todo[size=\small]{add reference to northward moving winter storm tracks and emissions.}

Despite possible recent intensification of freezing rain events, reports of freezing rain remain mostly ($>$80\%) of light intensity and ice accretion is chiefly a function of event duration. An indication that the archetypal pattern which causes the longest freezing rain events is becoming more prevalent is a sign that there could be real implications for decision-makers considering ice storm impacts. \todo[size=\small]{Tie in duration trends once finished with them}These findings of regional trends in the freezing rain climatology may inform next-generation hazard analyses that can help decision makers plan for future ice accretion thicknesses, such as the work done in \citet{erfani2014aggregated}.
\todo[inline,size=\small]{This summary needs work---basically all the quantified trends aren't included here. I need a paragraph or two explicitly stating where we've found trends, what they are, where we haven't, what it means.}

Opportunities exist for urban planners, power system operators, biologists, and other decision-makers to respond to the changing patterns of ice storms reported here and expected in the near future. Established resilience measures abound: borders of ice-loading districts determined by the National Electrical Safety Code Committee could be redefined in the future based upon updated ice recurrence intervals to save costs on new transmission lines where glaze ice accumulation has become less severe and improve reliability where ice storm impacts have intensified \citep{american2013minimum}; transportation safety can be improved by enhancing public awareness about local changes in ice storm climatology during freezing rain events \citep{call2009assessment}; power outages can sometimes be avoided by intentionally heating critical transmission lines with overcurrent \citep{bendel1981review,huneault2005combined}; wind turbine curtailment can be better accounted for in financial projections and electricity markets and can potentially be minimized using cameras and recent innovations in blade heating technologies \citep{bird2014wind}; buildings and natural environments could withstand more severe glaze ice accumulation if urban planners and forest managers plant tree species that are resistant to ice accretion and prioritize canopy-thinning efforts in fragile ecosystems \citep{hauer2006trees}.






%%%%%%%%%%%%%%%%%%%%%%%%%%%%%%%%%%%%%%%%%%%%%%%%%%%%%%%%%%%%%%%%%%%%%
% ACKNOWLEDGMENTS
%%%%%%%%%%%%%%%%%%%%%%%%%%%%%%%%%%%%%%%%%%%%%%%%%%%%%%%%%%%%%%%%%%%%%
%
\acknowledgments
The authors would like to thank the Great Lakes Integrated Sciences \& Assessments team for its support and feedback; Profs. Allison Steiner and Xianglei Huang for their crucial input; Benjamin Bernstein and John Cortinas for their expert feedback; Benjamin Mallernee for his early work on the project; and Jennifer Bukowski for her advice on synoptic weather typing. An implementation of the seasonal Kendall trend test modified for serial dependence was obtained from Jeff Burkey via the MATLAB exchange website. ASOS data from the Integrated Surface Database were provided by NCEI. NARR and CPC US Unified Precipitation data were provided by the NOAA/OAR/ESRL PSD, Boulder, Colorado, USA, via OPeNDAP at https://www.esrl.noaa.gov/psd/.  

%%%%%%%%%%%%%%%%%%%%%%%%%%%%%%%%%%%%%%%%%%%%%%%%%%%%%%%%%%%%%%%%%%%%%
% REFERENCES
%%%%%%%%%%%%%%%%%%%%%%%%%%%%%%%%%%%%%%%%%%%%%%%%%%%%%%%%%%%%%%%%%%%%%
% Make your BibTeX bibliography by using these commands:
\bibliographystyle{ametsoc2014}
\bibliography{references}


\end{document}

